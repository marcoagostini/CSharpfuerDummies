\usepackage[T1]{fontenc}
\usepackage[utf8]{inputenc}
\usepackage{lmodern}

\usepackage{geometry} 
\usepackage{graphicx}
\usepackage[german, english]{babel}
\usepackage{listings}
\usepackage{xcolor}
\usepackage{subfiles}
\usepackage{enumitem}
\usepackage{hyperref}
\usepackage{fancyhdr}

% compact items
\setlength{\parindent}{0cm}
\setlength{\parskip}{1ex}

\graphicspath{ {images/} }
\geometry{top=2cm,left=2cm,right=2cm,bottom=2cm} 

\pagestyle{fancy}
\rhead{.NET Technonlogien}
\lhead{ \leftmark  }

\definecolor{codegreen}{rgb}{0,0.6,0}
\definecolor{codegray}{rgb}{0.5,0.5,0.5}
\definecolor{codepurple}{rgb}{0.58,0,0.82}
\definecolor{backcolour}{rgb}{0.95,0.95,0.95}

\lstdefinestyle{sharpc}{
	language=csh,
    backgroundcolor=\color{backcolour},   
    commentstyle=\color{codegreen},
    identifierstyle=\color{blue},
    keywordstyle=\color{magenta},
    numberstyle=\tiny\color{black},
    stringstyle=\color{orange}\ttfamily,
    basicstyle=\ttfamily,
    breakatwhitespace=false,         
    breaklines=true,
    morecomment=[l]{//}, %use comment-line-style!
	morecomment=[s]{/*}{*/}, %for multiline comments                 
    captionpos=b,                    
    keepspaces=true,                 
    numbers=left,                    
    numbersep=2pt,                  
    showspaces=false,                
    showstringspaces=false,
    showtabs=false,                  
    tabsize=2,
    morekeywords={ abstract, event, new, struct,
	as, explicit, null, switch,
	base, extern, object, this,
	bool, false, operator, throw,
	break, finally, out, true,
	byte, fixed, override, try,
	case, float, params, typeof,
	catch, for, private, uint,
	char, foreach, protected, ulong,
	checked, goto, public, unchecked,
	class, if, readonly, unsafe,
	const, implicit, ref, ushort,
	continue, in, return, using,
	decimal, int, sbyte, virtual,
	default, interface, sealed, volatile,
	delegate, internal, short, void,
	do, is, sizeof, while,
	double, lock, stackalloc,
	else, long, static,
	enum, namespace, string},
}

\lstset{style=sharpc}

\newcommand{\SubItem}[1]{
	{\setlength\itemindent{15pt} \item[-] #1}
}