\section{Exceptions}

\begin{lstlisting}
try
{
    d3 = Convert.ToDouble("12355555555555555555555555555555555555");
}
catch (FormatException ex)
{
    Console.WriteLine($"{ex.GetType().FullName} occured.");
}
catch (OverflowException ex)
{
    Console.WriteLine($"{ex.GetType().FullName} occured.");
}
catch (Exception ex)
{
    Console.WriteLine($"{ex.GetType().FullName} occured.");
}
\end{lstlisting}


\section{Iteratoren}

\subsection{ForEach}
Der ForEach Loop ist ein reines Compiler-Feature. Benötigt die Implementation von \lstinline{IEnumerator, IEnumerable} auf der Klasse.
\begin{lstlisting}
foreach (ElementType elem in collection){ statement };
\end{lstlisting}

\begin{lstlisting}
int[] list = { 1, 2, 3, 4, 5, 6 }; 
foreach (int i in list) {
	if (i == 3) continue; 
	if (i == 5) break; 
	Console.WriteLine(i);
}
// Compiler Output
IEnumerator enumerator = list.GetEnumerator(); 
try {
	while (enumerator.MoveNext()) {
		int i = (int)enumerator.Current; 
		if (i == 3) continue; 
		if (i == 5) break; 
		Console.WriteLine(i);
	}
} finally {
	IDisposable disposable = enumerator as IDisposable;
	if (disposable != null) {
		disposable.Dispose();
	}
}
\end{lstlisting}


\subsection{Enumartion}
\textbf{IEnumerable:} ist ein Interface, das die Methode GetEnumerator definiert, dieser hat den Typ IEnumerator.
\textbf{IEnumerator:} ist ein Interface, welches über die Elemente einer Collection iteriert Funktionen: \lstinline{ MoveNext(), Reset()} enthält.

\begin{lstlisting}
CityCollection myColl = new CityCollection();

foreach (string s in myColl)
{
    Console.WriteLine(s);
}
// Compiler output
var enumerator = myColl.GetEnumerator();
while (enumerator.MoveNext())
{
    Console.WriteLine(enumerator.Current);
}

// Definition der Collection
public class CityCollection : IEnumerable<string>
{
    private string[] cities = { "Bern", "Basel", "Zuerich", "Rapperswil"};
    public IEnumerator<string> GetEnumerator()
    {
        for (int i = 0; i < cities.Length; i++) yield return cities[i];
    }
    IEnumerator IEnumerable.GetEnumerator()
    {
        return GetEnumerator();
    }
}
\end{lstlisting}

\subsection{Extension Methods}

\begin{lstlisting}
public static class Extensions
{
    public static IEnumerable<int> OstMultipleOf(this IEnumerable<int> source, int factor)
    {
        foreach (int item in source)
            if (item % factor == 0) yield return item;      
    }
}

int[] numbers = { 1, 4, 2, 9, 13, 8, 9, 0, -6, 12 };
IEnumerable<int> multipleof4 = numbers.OstMultipleOf(4);
foreach (int i in multipleof4) { Console.WriteLine(i); }
\end{lstlisting}








