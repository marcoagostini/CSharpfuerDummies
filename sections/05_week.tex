\section{Delegates}
Ein Delegate verbindet einen Aufrufer zur Laufzeit mit seiner Zielmethode.

\begin{lstlisting}
delegate int Comparer(object x, object y);
class Car
{
    public string Brand { get; }
    public int EngineSize { get; set; }
    public int WheelSize { get; set; }

    public Car (string Brand, int EngineSize, int WheelSize)
    {
        this.Brand = Brand;
        this.EngineSize = EngineSize;
        this.WheelSize = WheelSize;
    }

    public static int CompareEngine(object x, object y)
    {
        Car c1 = (Car)x;
        Car c2 = (Car)y;
        if (c1.EngineSize < c2.EngineSize) return -1;
        else if (c1.EngineSize > c2.EngineSize) return 1;
        else return 0;
    }

    public static void CompareCar(Car x, Car y, Comparer compare)
    {
        int result = compare(x, y);
        Console.WriteLine(result);
    }
}

class Program
{
    static void Main(string[] args)
    {
        Car c1 = new Car("Ferrari", 4, 20);
        Car c2 = new Car("Lamborghini", 12, 20);
        Car.CompareCar(c1, c2, Car.CompareEngine);
    }
}
\end{lstlisting}


\subsection{Multicast Delegate}
Jedes Delegate ist auch ein Multicast Delegate.
\begin{lstlisting}
public delegate void Notifier(string Person);

class Person
{
    public string Name { get; set; }
    public int Age { get; set; }

    public static void sayHi(string sender)
    {
        Console.WriteLine("Hello {0}", sender);
    }

    public static void sayCiao(string sender)
    {
        Console.WriteLine("Ciao {0}", sender);
    }
}
static void Main(string[] args)
{
    Notifier n1 = sayHi;
    n1 += sayCiao;
    n1 += sayCiao;
    n1.Invoke("Marco Agostini");
}
\end{lstlisting}


\section{Events}
\begin{lstlisting}
   // 1. Define Delegate
   public delegate void RaceEventHandler(object source);

   // 2. Define Publisher
   public class RaceController
   {
      // 3. Define an event based on Delegate
      public event RaceEventHandler RaceChangeEvent;
      public void ChangeRaceState()
      {
         RaceChangeEvent?.Invoke(this);
      }
   }
    
   // 5. Write Subscribers
   public class Car
   {
      public int number { get; }
      public bool IsRunning { get; set; } = false;
      public Car(int number) { this.number = number; }
      public void ChangeCarState(object source)
      {
         if (!IsRunning)
         {
            IsRunning = true;
            Console.WriteLine("Car with number {0} has started", number);
         } else {
            IsRunning = false;
            Console.WriteLine("Car with number {0} has stopped", number);
         }
      }
   }

   public static void Main(string[] args)
   {
      RaceController rh1 = new RaceController();
      Car c1 = new Car(1);
      Car c2 = new Car(2);

      // Add Subscribers to Event
      rh1.RaceChangeEvent += c1.ChangeCarState;
      rh1.RaceChangeEvent += c2.ChangeCarState;
      
      rh1.ChangeRaceState();
      rh1.ChangeRaceState();

      rh1.RaceChangeEvent -= c1.ChangeCarState;
      rh1.RaceChangeEvent -= c2.ChangeCarState;
   }
\end{lstlisting}